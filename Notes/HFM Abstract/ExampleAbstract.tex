%%%%%%%%%%%%%%%%%%%%%%%%%%%%%%%%%%%%%
%DO NOT EDIT THIS SECTION
\documentclass[12pt,english]{article}
\usepackage[latin1]{inputenc}
\usepackage{geometry}
\geometry{verbose,tmargin=0.75in,bmargin=0.75in,lmargin=1in,rmargin=1in}
\pagestyle{empty}
\setlength\parskip{\medskipamount}
\setlength\parindent{0pt}
\usepackage{float}
\usepackage{graphicx}
\usepackage{epstopdf}
\makeatletter
\usepackage{babel}
\makeatother
\begin{document}
%%%%%%%%%%%%%%%%%%%%%%%%%%%%%%%%%%%%%
Abstract ID % Do not edit this line
\begin{center}{  % Do not edit this line
\large\bf % Do not edit this line
% Title of the abstract:
The hitchhiker's guide to the kagome lattice
} % Do not edit this line
 % Do not edit this line
\bigskip % Do not edit this line


\textbf{Subject area:}
%subject area:  uncomment one of the following
%%%%%%%%%%%%%%%%%%%%%%%%
%Pyrochlore systems
%Kagome systems
%triangular lattice systems
%Hyper-kagome systems
%Face-centred Cubic Systems
%Competing interactions on bi-partite lattices
%Artificial and molecular frustrated spin systems
%Spin glasses and random magnets
%Itinerant frustrated systems and exotic superconductivity
%Coupling between lattice, spin, orbital, and charge degrees of freedom.
%Multiferroic phenomena
%Other


% Presenting author:
\underbar{Arthur Dent},$^a$
% Coauthors (if any)
Ford Prefect,$^{a,b}$ 
Zaphod Beeblebrox,$^b$ and 
Tricia M. McMillan$^a$ 
\\  % Do not edit this line
\textit{ % Do not edit this line
% Affiliation a:
$^a$Department of Physics, 
Major Research University,
Guildford, Surrey GU2 7XH, UK
% Affiliation b:
\\ $^b$Big Government Lab, 
Small Planet, Betelgeuse, Orion
} % Do not edit this line
\end{center}

\begin{figure}[H]  
\begin{center}  
% Your figure in EPS or PDF:
\includegraphics[width=6cm]{DentFig1.eps}
\end{center} 
\begin{center}  
% Figure caption:
\textbf{Fig. 1.}  A figure and its caption.  
The EPS file used in it is named DentFig1.eps. 
\end{center}  
\end{figure}  
 
% Begin main text: 

The Japanese word \textit{kagome} means a bamboo-basket (\textit{kago}) woven pattern (\textit{me}) that is composed of interlaced triangles whose sites have four nearest neighbors [1].  This lattice was first introduced into statistical physics by Kodi Husimi at Osaka University.  Husimi's research associate Itiro Sy\^ozi studied the Ising ferro and antiferromagnets on this lattice [2].  

In the late 1980s, Veit Elser initiated studies of the $S=1/2$ Heisenberg antiferromagnet on kagome [3].  This system is a prime candidate for an unusual ground state without long-range magnetic order and with exotic magnetic excitations [4].  It may finally realize Phil Anderson's proposal of a quantum state with resonating valence bonds [5], originally made for the triangular lattice.  A snapshot of such a state on kagome is shown in Fig. 1.

Successful synthesis and characterization of $S=1/2$ antiferromagnets with this sort of spin arrangements [6] promises some serious fun for both experimentalists and theorists in the near future.

% Acknowledgments of colleagues and funding agencies:

We acknowledge useful discussions with P. A. Marvin.
This research was supported in part by the NSF Grant No. $3.1415926\ldots$

\bigskip  % Do not edit this line

% References: 
$[$1$]$ M. Mekata, Physics Today \textbf{56,} No. 2, 12 (2003).\\
$[$2$]$ I. Sy\^ozi, Prog. Theor. Phys. \textbf{6,} 306 (1951).\\
$[$3$]$ V. Elser, Phys. Rev. Lett. \textbf{62,} 2405 (1989).\\
$[$4$]$ P. A. Lee, Science \textbf{321,} 1306 (2008).\\
$[$5$]$ P. W. Anderson, Mat. Res. Bull. \textbf{8,} 153 (1973).\\
$[$6$]$ Z. Hiroi \textit{et al.}, J. Phys.: Conf. Ser. \textbf{145,} 012002 (2009).

\end{document}  % Do not edit this line
