
%%%%%%%%%%%%%%%%%%%%%%%%%%%%%%%%%%%%%
%DO NOT EDIT THIS SECTION
\documentclass[12pt,english]{article}
\usepackage[latin1]{inputenc}
\usepackage{geometry}
\geometry{verbose,tmargin=0.75in,bmargin=0.75in,lmargin=1in,rmargin=1in}
\pagestyle{empty}
\setlength\parskip{\medskipamount}
\setlength\parindent{0pt}
\usepackage{float}
\usepackage{graphicx}
\usepackage{epstopdf}
\makeatletter
\usepackage{babel}
\makeatother
\begin{document}
%%%%%%%%%%%%%%%%%%%%%%%%%%%%%%%%%%%%%
Abstract ID % Do not edit this line
\begin{center}{  % Do not edit this line
\large\bf % Do not edit this line
% Title of the abstract:
DMRG Study of the $S>1/2$ quantum Heisenberg Antiferromagnet on 
a Kagome-like lattice without loops
%%% the Husimi Cactus
} % Do not edit this line
 % Do not edit this line
\bigskip % Do not edit this line


\textbf{Subject area:}
%subject area:  uncomment one of the following
%%%%%%%%%%%%%%%%%%%%%%%%
%Pyrochlore systems
Kagome systems
%triangular lattice systems
%Hyper-kagome systems
%Face-centred Cubic Systems
%Competing interactions on bi-partite lattices
%Artificial and molecular frustrated spin systems
%Spin glasses and random magnets
%Itinerant frustrated systems and exotic superconductivity
%Coupling between lattice, spin, orbital, and charge degrees of freedom.
%Multiferroic phenomena
%Other


% Presenting author:
\underbar{R. Zach Lamberty},$^a$
% Coauthors (if any)
Hitesh Changlani,$^a$ and 
Christopher L. Henley$^a$ 
\\  % Do not edit this line
\textit{ % Do not edit this line
% Affiliation a:
$^a$Department of Physics, 
Cornell University
Ithaca, New York, 14850, USA
} % Do not edit this line
\end{center}

%\begin{figure}[H]  
%\begin{center}  
%% Your figure in EPS or PDF:
%\includegraphics[width=6cm]{DentFig1.eps}
%\end{center} 
%\begin{center}  
%% Figure caption:
%\textbf{Fig. 1.}  A figure and its caption.  
%The EPS file used in it is named DentFig1.eps. 
%\end{center}  
%\end{figure}  
 
% Begin main text: 

We consider the quantum Heisenberg antiferromagnet 
for $S\ge1/2$ on the Husimi cactus, a graph of corner sharing triangles
each of whose centers is a vertex of a Bethe lattice.
%derived from a Bethe lattice
%in which each vertex is shared by two triangles.
Our focus is the possible transition(s) as $S$ is increased from an
expected spin liquid at $S=1/2$ to 
the coplanar ordered state known to be stable in the large $S$ limit.
[P. Simon and B. Doucot, c 1999 -- FILL IN?]
Since the geometry is like the Kagome lattice locally, 
properties dominated by nearest-neighbor spin correlations
should be captured by the same interactions on the Husimi cactus;
on the other hand, the cactus lacks loops, so properties dependent on them
cannot be captured.

Our method is a DMRG procedure tailored for tree graphs,
well suited to our models;
by contrast, such $S>1/2$ models are almost intractable by 
exact diagonalization on the kagome lattice.


% Acknowledgments of colleagues and funding agencies:

This research was supported in part by NSF Grant DMR$-1005466$ 
and by an NSF Graduate Research Fellowship to R. Zach Lamberty.

\bigskip  % Do not edit this line

% References: 
$[$1$]$ Some Dude, Some Journal \textbf{99,} No. 9, 99 (9999).\\

\end{document}  % Do not edit this line
