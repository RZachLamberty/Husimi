%%%%%%%%%%%%%%%%%%%%%%%%%%%%%%%%%%%%%
%DO NOT EDIT THIS SECTION
\documentclass[12pt,english]{article}
\usepackage[latin1]{inputenc}
\usepackage{geometry}
\geometry{verbose,tmargin=0.75in,bmargin=0.75in,lmargin=1in,rmargin=1in}
\pagestyle{empty}
\setlength\parskip{\medskipamount}
\setlength\parindent{0pt}
\usepackage{float}
\usepackage{graphicx}
\usepackage{epstopdf}
\makeatletter
\usepackage{babel}
\makeatother
\begin{document}
%%%%%%%%%%%%%%%%%%%%%%%%%%%%%%%%%%%%%
Abstract ID % Do not edit this line
\begin{center}{  % Do not edit this line
\large\bf % Do not edit this line
% Title of the abstract:
DMRG Study of the $S>1/2$ quantum Heisenberg Antiferromagnet on the Husimi Cactus
} % Do not edit this line
 % Do not edit this line
\bigskip % Do not edit this line


\textbf{Subject area:}
%subject area:  uncomment one of the following
%%%%%%%%%%%%%%%%%%%%%%%%
%Pyrochlore systems
Kagome systems
%triangular lattice systems
%Hyper-kagome systems
%Face-centred Cubic Systems
%Competing interactions on bi-partite lattices
%Artificial and molecular frustrated spin systems
%Spin glasses and random magnets
%Itinerant frustrated systems and exotic superconductivity
%Coupling between lattice, spin, orbital, and charge degrees of freedom.
%Multiferroic phenomena
%Other


% Presenting author:
\underbar{R. Zach Lamberty},$^a$
% Coauthors (if any)
Hitesh Changlani,$^a$ and 
Christopher L. Henley$^a$ 
\\  % Do not edit this line
\textit{ % Do not edit this line
% Affiliation a:
$^a$Department of Physics, 
Cornell University
Ithaca, New York, 14850, USA
} % Do not edit this line
\end{center}

%\begin{figure}[H]  
%\begin{center}  
%% Your figure in EPS or PDF:
%\includegraphics[width=6cm]{DentFig1.eps}
%\end{center} 
%\begin{center}  
%% Figure caption:
%\textbf{Fig. 1.}  A figure and its caption.  
%The EPS file used in it is named DentFig1.eps. 
%\end{center}  
%\end{figure}  
 
% Begin main text: 

Using a DMRG procedure tailored for tree graphs, we consider the quantum Heisenberg antiferromagnet on the Husimi cactus for $S>1/2$.  The geometry of the Husimi cactus replicates the Kagome lattice locally, but contains no loops.  Properties of the Kagome lattice which are dominated by nearest-neighbor spin fluctuations should therefore be captured adequately by the same interactions on the Husimi cactus.  Comparison with previous results for the Kagome lattice may shed light on how much of the physics is determined by nearest-neighbor interactions, and how much is dependent on loop corrections.  We also investigate whether or not there is a transition from a possible small-$S$ spin liquid state to the known coplanar state at large $S$, a problem which is prohibitively difficult on the regular Kagome lattice.

% Acknowledgments of colleagues and funding agencies:

This research was supported in part by the NSF Grant DMR$-1005466$ and an NSF GRF for R. Zach Lamberty.

\bigskip  % Do not edit this line

% References: 
$[$1$]$ Some Dude, Some Journal \textbf{99,} No. 9, 99 (9999).\\

\end{document}  % Do not edit this line
