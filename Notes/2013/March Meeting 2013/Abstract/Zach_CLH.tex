
%%%%%%%%%%%%%%%%%%%%%%%%%%%%%%%%%%%%%%%%%%%%%%%%%%%%%%
%%                                                  %%
%%  March Meeting 2013 Abstract (second draft)      %%
%%                                                  %%
%%      Zach Lamberty                               %%
%%      Nov. 5, 2012                                %%
%%                                                  %%
%%%%%%%%%%%%%%%%%%%%%%%%%%%%%%%%%%%%%%%%%%%%%%%%%%%%%%


% set the document class to that of revtex4-1 with the general option AIP.
% Other options:
%   nofootinbib --  Keeps footnotes on pages they are located and doesn't try to put them in the bib.  Leaves bib items alone.
%%% \documentclass[aps, superscriptaddress]{revtex4-1}
\documentclass[aps]{revtex4-1}


% Open any necessary packages.
\usepackage{amsfonts}           % AMS packages
\usepackage{amssymb}            % ^
\usepackage{amsmath}            % ^
\usepackage{latexsym}           % For the "normal" symbol (and other funky binary operators)
\usepackage{dcolumn}            % Align table columns on decimal point
%\usepackage[dvips]{graphicx}   % For production with eps / jpg figures
\usepackage{enumerate}
\usepackage{epstopdf}           % Can incorporate eps in the above pdftex
\usepackage{fancyhdr}           % Does exactly what it says it will
\usepackage[pdftex]{graphicx}   % For production with pdf figures
\usepackage{hyperref}           % Allow for hyperlinks
\usepackage{setspace}           % Allows us to change spacings
\usepackage{xspace}             % To include spaces after newly defined commands when they are expecting arguments
\usepackage{subfigure}          % Gives us the ability to put several figures side-by-side within one.
\usepackage{bm}                 % A better "bold math" package -- any math symbol can be emboldened with the command \bm{*}
%\usepackage[table]{xcolor}      % Alternating colors for table rows via the \rowcolors{<starting row>}{<odd color>}{<even color>} command
\usepackage[usenames]{color}      % Color text
\usepackage{multirow}           %   Have column entries in tables span multiple rows

% Chose our headings.  Some options:
%   plain:      empty header, p. on the bottom center.  Default
%   headings:   empty footer, current chapter & p. no. in the header
%   empty:      both empty.
%   fancy:      Left, center, and right options.  Use package "fancyhdr" with commands \*head{text}, * = l,c,r
\pagestyle{plain}

%setting up our fancy header
%\lhead{Homework 6 Solutions}
%\chead{Zach Lamberty}
%\rhead{Physics 1203, Spring 2010}

%sets indent on the paragraph
%\setlength{\parindent}{-.05in}

%set the spacing
%\onehalfspacing

%set our margins for single sided
%\setlength{\oddsidemargin}{.25in}
%\setlength{\evensidemargin}{.25in}

%set our textwidth
%\setlength{\textwidth}{6in}
%\setlength{\headwidth}{6in}

% Define any new commands
\newcommand{\bp}[1]{\textbf{ Problem #1} \\}
\newcommand{\ud}{\text{ m}}
\newcommand{\uv}{\text{ m/s}}
\newcommand{\ua}{\text{ m/${\text{s}}^2$}}
\newcommand{\ut}{\text{ s}}
\newcommand{\bZ}{\mathbb{Z}}
\newcommand{\br}{\mathbf{r}}


\begin{document}

%Title.  Two "affiliation" class options are groupedaddress (default) and superscriptaddress
\title{DMRG Study of the $S\ge 1$ quantum Heisenberg Antiferromagnet on a Kagome-like lattice without loops}
\date{\today}

\author{\firstname{R. Zach} Lamberty}
\author{Hitesh Changlani}
\author{\firstname{C. L.} Henley}
\affiliation{Cornell University, Ithaca, New York, 14853}

\begin{abstract}
The Kagome quantum Heisenberg antiferromagnet, not only for spin $S=1/2$ 
but for $S=1$ and perhaps $S=3/2$,  is a prime candidate to realize a 
quantum spin liquid or valence bond crystal state, but 
theoretical and computational studies for $S>1/2$ are difficult and few.
To address the latter regime, we consider the quantum Heisenberg 
antiferromagnet for $S\ge1/2$ on the Husimi cactus, a graph of corner sharing 
triangles each of whose centers is a vertex of a Bethe lattice, 
using a DMRG procedure tailored for tree graphs $[$1$]$.
Since the geometry is like the Kagome lattice locally, 
properties dominated by nearest-neighbor spin correlations 
should be captured by the same interactions on the Husimi cactus; 
on the other hand, since the cactus lacks loops, 
properties dependent on them cannot be captured.  
The cactus antiferromagnet is known to have a disordered valence bond state
at $S=1/2$ but a three-sublattice coplanar ordered state in the large 
$S$ limit $[$2$]$. 
Thus, our focus is the possible transition(s) that must occur with increasing 
$S$.  As a test of this approach, we also investigate the phase diagram 
of the $S=1$ quantum XXZ model with on-site anisotropy, which we expect to
have a variety of phases (such as three-sublattice order, valence bond crystal)
similar to the kagome case $[$3$]$.

This work is supported by the National Science Foundation through a Graduate Research Fellowship to R. Zach Lamberty, as well as grant DMR-0552461.\\
\\
\noindent
% References: 
$[$1$]$ H. J. Changlani, S. Ghosh, C. L. Henley, A. M. L\"{a}uchli, arXiv:1208.1773v1 (2012).\\
$[$2$]$ B. Doucot and P. Simon, J. Phys. A \textbf{31}, 5855 (1998).\\
$[$3$]$ S. V. Isakov and Y. B. Kim, Phys. Rev. B \textbf{79}, 094408 (2009).\\

\end{abstract}
\maketitle

\end{document}
